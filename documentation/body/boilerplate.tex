\newpage
\chapter*{Boilerplate}
\addcontentsline{toc}{chapter}{Boilerplate}
% Chapter for each research question
% how we validated it and proved it worked


Boilerplate's are a pre-designed webpages, or a set of HTML webpages that an end user can customising by adding in their own imagery, and content. The boilerplates include all the files such as the HTML, CSS, and JavaScript files required for the boilerplate to run smoothly. 

Boilerplate's simplify the web development process, by making it easy for the end user who have little or no programming experience to build their own websites.

Taylor'd UI offers three boilerplate templates for the end user to utilise in their learning of CSS; a blog, portfolio, and product website. 

While each of the boilerplates utilise the framework as their foundations. Each boilerplates also have their own css files with code that is unique to the boilerplate such as the blog boilerplate has a Read More button that expands the accompanying text section. The portfolio boilerplate has text that appears as an overlay when the user hovers over the image.

The blog template follows a simple theme, a jumbotron that has the title, and subtitle of the blog, followed by a blog post. The blog post has a title, and the date published at the top followed by an image. All the placeholder imagery is hosted by placehold.it \citep{PHI17}. 

By using an online image hosting server, the boilerplate files are smaller in size, links to the images are also less likely to break. Underneath the image, a paragraph of text is displayed, the user can click on the read more / read less button for more or less text to appear. The blog publisher and number of comments is the last section of the post. 