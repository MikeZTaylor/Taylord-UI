\newpage
\chapter*{Taylor'd UI}
\addcontentsline{toc}{chapter}{Taylor'd UI}
% Chapter for each research question
% how we validated it and proved it worked

Taylor'd UI is a full \gls{UI} framework that is aimed towards an entry level user. The framework is non-opinionated, light weight, easy to understand, theme-able, and most importantly easy to learn. 

The vocabulary used in Taylor'd UI has been designed to be user friendly so that when a user reads the syntax, they should have an idea of what the end result will look like without writing any code. The vocabulary follows english terminology, for example a pill shaped buttons syntax is: 

\begin{lstlisting}[language=HTML]
<button class="pill-shaped-button">Button</button>
\end{lstlisting}

Taylor'd UI has minimal styling, the author has only added styling where it is needed, and that styling is kept to basic colours. This was to ensure the framework does not carry the author's design opinions. Being non-opinionated, Taylor'd UI is intensionally bare so that the end user is encouraged to not rely upon the built-in styling, but instead to use it as a starting point to build upon it. 

One of the aims is to make the library as small as possible to ensure it loads quickly on mobile devices and in situations where fast network connectivity is not available. This also ensures that the data used on mobile devices is not unnecessarily burdened.

To stop code from being repeated unnecessarily, code sections have been broken into partials to allow for the ability for the content to be broken up into manageable pieces, removing receptive code such as headers, and footers.

Taylor'd UI is made up of the following features; typography, buttons, alerts, tables, panels, jumbotron, button groups, forms, and a responsive 12 grid system.

\newpage
\subsection*{Typography}
\addcontentsline{toc}{subsection}{Typography}
The typography used in Taylor'd UI is Open Sans. The default font size throughout the document is 16. There are text modifiers such as extra-small-p, and extra-large-p that decrease or increase the font size, a visual can be seen on Figure.~\ref{fig:fontSize} on  page~\pageref{fig:fontSize}. An example on how to call these classes: 

\begin{lstlisting}[language=HTML]
<p class="extra-small-p">Text</p>      
<p class="small-p">Text</p>      
<p class="large-p">Text</p>       
<p class="extra-large-p">Text</p>
\end{lstlisting}

The text can be aligned; left aligned, right aligned or justified. These classes can be called directly from the paragraph tag. A visual can be seen on Figure.~\ref{fig:textalign} on  page~\pageref{fig:textalign}. An example on how to call these classes: 

 \begin{lstlisting}[language=HTML]
<p class="left-align-text"></p>
<p class="right-align-text"></p>
<p class="justified-text"></p>
\end{lstlisting}

There are four different types of font weightings as seen on Figure.~\ref{fig:fontweight} on  page~\pageref{fig:fontweight}. these font weightings can be used on paragraph, label, and heading texts. These weightings range from thin to bold.

\begin{lstlisting}[language=HTML]
<p class="thin-p">Text</p>
<p class="bold-p">Text</p>
\end{lstlisting}

The default font weighting for the paragraph text is 400. To change the weighting of the font, you can change the number value at the end of the variable. To save confusion, it would also be a good idea to change the variable name as seen below.

\begin{lstlisting}[language=HTML]
$font-weight-400: 400; //default
\end{lstlisting}

\begin{lstlisting}[language=HTML]
$font-weight-400: 700; //bolder
\end{lstlisting}

\newpage
\subsection*{Buttons}
\addcontentsline{toc}{subsection}{Buttons}

There are four button types in Taylor'd UI as seen in Figure.~\ref{fig:genButtons} on  page~\pageref{fig:genButtons}. These button types are styled in a similar way to how a browser would display a button. Below is the syntax used to call these classes: 

\begin{lstlisting}[language=HTML]
 <button class="button">Test Button</button>
<button class="large-button">Test Button</button>
<button class="small-button">Test Button</button>
<button class="pill-shaped-button">Test Button</button>
\end{lstlisting}

Additionally, a button can be styled using modifier classes. These modifiers are based on the primary colours as seen in Figure.~\ref{fig:buttonMods} on  page~\pageref{fig:buttonMods}. These colours are colours that a user would have seen in other applications used for alerts such warning, and success actions. The code snippet below shows the different types.

\begin{lstlisting}[language=HTML]
<button class="button default-button">default</button>
<button class="button primary-button">primary</button>
<button class="button success-button">success</button>
<button class="button info-button">Info</button>
<button class="button warning-button">warning</button>
<button class="button danger-button">danger</button>
<button class="button link-button">button Link</button>
\end{lstlisting}

The button modifiers can be used with the generic button classes. An example on how to call these classes: 

\begin{lstlisting}[language=HTML]
<button class="large-button info-button"></button>
<button class="pill-shaped-button warning-button"></button>
\end{lstlisting}

\newpage
\subsection*{Alerts}
\addcontentsline{toc}{subsection}{Alerts}
Alerts provide user feedback based on certain actions performed by them. An alert is built using the .alert class, and styled with modifier classes such as seen in Figure.~\ref{fig:alerts} on  page~\pageref{fig:alerts}. The code snippet below shows an example.

\begin{lstlisting}[language=HTML]
<div class="alert info-alert">
<p>Lorem ipsum dolor sit ame</p>
<div class="alert success-alert">
<p>Lorem ipsum dolor sit ame</p>
so please pay attention.</p></div>
\end{lstlisting}

An alert class can be built upon by adding buttons, see Figure.~\ref{fig:AlertModifiers} on  page~\pageref{fig:AlertModifiers} as well as links. The code snippet below shows an example.
 
\begin{lstlisting}[language=HTML]
<div class="alert alert-primary">
<p>Lorem ipsum dolor sit ame</p>
<button class="button button-primary">Do Some Action</button>
 <button class="button button-link">Cancel</button>
\end{lstlisting}

\subsection*{Tables}
\addcontentsline{toc}{subsection}{Tables}
A table allow web authors to arrange data such as text, images, links, even other tables into rows and columns of cells. The tables are created using the table tag in which the tr tag is used to create table rows and td tag is used to create data cells.

A default table was created for Taylor'd UI. The default table is simple, with clear emphasis on the headings of the table. 

\begin{lstlisting}[language=HTML]
<table class="table">
	<thead>
		<tr>
			<th>Lorem ipsum dolor sit ame</th>
			<th>consectetur adipiscing elit</th>
			<th>Quasi vero</th>
		</tr>
	</thead>
	<tbody>
		<tr>
			<td>inquit</td>
			<td>perpetua oratio rhetorum solum</td>
			<td>non etiam philosophorum sit</td>
		</tr>
		<tr>
			<td>magna dissensio</td>
			<td>In contemplatione et cognitione</td>
			<td>Num igitur dubium es</td>
		</tr>
		<tr>
			<td>Non est ista</td>
			<td>inquam</td>
			<td>Piso</td>
		</tr>
	</tbody>
</table>
\end{lstlisting}

There are four modifiers see  Figure.~\ref{fig:tableMod} on  page~\pageref{fig:tableMod} that can be added to the table class for a different look as seen below. The other modifiers are hover-on-table,  bordered-table. Below is an example of how to call one of the modifiers. 

\begin{lstlisting}[language=HTML]
<table class="table striped-table">
	<thead>
		<tr>
			<th>Lorem ipsum dolor sit ame</th>
			<th>consectetur adipiscing elit</th>
			<th>Quasi vero</th>
		</tr>
	</thead>
	<tbody>
		<tr>
			<td>inquit</td>
			<td>perpetua oratio rhetorum solum</td>
			<td>non etiam philosophorum sit</td>
		</tr>
		<tr>
			<td>magna dissensio</td>
			<td>In contemplatione et cognitione</td>
			<td>Num igitur dubium es</td>
		</tr>
		<tr>
			<td>Non est ista</td>
			<td>inquam</td>
			<td>Piso</td>
		</tr>
	</tbody>
</table>
\end{lstlisting}

Modifiers from other components such as alerts can be used on table elements, for a visual see Figure.~\ref{fig:tableClass} on  page~\pageref{fig:tableClass}. The code snippet below shows you how to call the different modifiers. 

\begin{lstlisting}[language=HTML]
<table class="table">
	<thead>
		<tr>
			<th>Modifiers</th>
		</tr>
	</thead>
	<tbody>
		<tr class="is-success">
			<td>success</td>
		</tr>
		<tr class="is-primary">
			<td>primary</td>
		</tr>
		<tr class="is-info">
			<td>info</td>
		</tr>
		<tr class="is-warning">
			<td>warning</td>
		</tr>
		<tr class="is-danger">
			<td>danger</td>
		</tr>
	</tbody>
</table>
\end{lstlisting}

\newpage
\subsection*{Panels}
\addcontentsline{toc}{subsection}{Panels}
A panel in Taylor'd UI is a bordered box with some padding around its content. The basic panel is just that see Figure.~\ref{fig:panel} on  page~\pageref{fig:panel}. Other attributes such as heading tags can be added to a panel as seen below: 

\begin{lstlisting}[language=HTML]
<div class="panel default-panel">
	<div class="panel-title">Lorem Ipsumn
	</div>
	<div class="panel-body"> consectetur adipiscing elit, 
	sed do eiusmod tempor incididunt ut labore et dolore magna aliqu
	</div>
</div>
\end{lstlisting}

Modifiers from other components such as alerts see Figure.~\ref{fig:panelmod} on  page~\pageref{fig:panelmod} can be added to a panel. The code snippet below shows you one example: 

\begin{lstlisting}[language=HTML]
<div class="panel panel-info">
	<div class="panel-title">orem Ipsum</div>
	<div class="panel-body"> consectetur adipiscing elit, 
	sed do eiusmod tempor incididunt ut labore et dolore magna aliqu
	</div>
</div>
\end{lstlisting}


\subsection*{Labels}
\addcontentsline{toc}{subsection}{Labels}

Labels are built with button-label class, the default label does not need the button class, as seen below: 

\begin{lstlisting}[language=HTML]
<span class="button-label">Default Label</span>
\end{lstlisting}

Modifiers can be added to the label for more impact Figure.~\ref{fig:label} on  page~\pageref{fig:label}. The below code is all of the label classes:

\begin{lstlisting}[language=HTML]
<span class="button-label">Default Label</span>
<span class="button-label primary">Default Primary</span>
<span class="button-label success">Success Label</span>
<span class="button-label info">Info Label</span>
<span class="button-label warning">Warning Label</span>
<span class="button-label danger">Danger Label</span>
\end{lstlisting}

\newpage
\subsection*{Jumbotron}
\addcontentsline{toc}{subsection}{Jumbotron}
Jumbotrons can be treated as large panels, the default panel has the text entered in the centre with lots of padding all round, as seen in Figure.~\ref{fig:jumbo} on  page~\pageref{fig:jumbo}. Below is the code snippet on how to create a jumbotron: 

\begin{lstlisting}[language=HTML]
<div class="jumbotron"><h1>Lorem Ipsum</h1></div>
\end{lstlisting}

As jumbotrons share similar attributes to panels, the same modifiers that apply to panels can also be applied here. 

\begin{lstlisting}[language=HTML]
<div class="jumbotron alert-primary"><h1>Lorem Ipsum</h1></div>
\end{lstlisting}

\subsection*{Button Groups / Pagination}
\addcontentsline{toc}{subsection}{Button Groups / Pagination}
Button groups are a collection of buttons on the same line. Button groups work in the same way that buttons work. To group a collection of buttons, add the class button-group around the collection as seen in Figure.~\ref{fig:defualtgroup} on  page~\pageref{fig:defualtgroup}. Below is an example of the default button group: 

\begin{lstlisting}[language=HTML]
<div class="button-group"><button class="button button-default">
Button One</button>
<button class="button default-button">Button Two</button>
<button class="button default-button">Button Three</button>
</div>
\end{lstlisting}

As with buttons, the same modifiers, see Figure.~\ref{fig:groupSize} on  page~\pageref{fig:groupSize} can be added to either increase the button-group size. The code snippet below shows you the large button group:

\begin{lstlisting}[language=HTML]
<div class="button-group large-button-group">
	<button class="button default-button">Button One</button>
	<button class="button default-button">Button Two</button>
	<button class="button default-button">Button Three</button>
</div>
\end{lstlisting}

Or adding colour to the group see Figure.~\ref{fig:modButtonGroup} on  page~\pageref{fig:modButtonGroup}. Below is one example of modifiers:  

\begin{lstlisting}[language=HTML]
<div class="button-group">
<button class="button primary-button">Button One</button>
<button class="button primary-button">Button Two</button>
<button class="button primary-button">Button Three</button>
</div>
\end{lstlisting}

\subsection*{Forms}
\addcontentsline{toc}{subsection}{Forms}
There are multiple elements that make up a form. Taylor'd UI caters for all these elements. The overall styling for the form comes from the form-styling class as seen below:
\begin{lstlisting}[language=HTML]
<div class="form-styling"></div>
\end{lstlisting}

The following inputs make the overall form see Figure.~\ref{fig:form} on  page~\pageref{fig:form}. Passwords are not entered in as plain text as the type has been set to passwords, allowing for asterisks to appear instead. All form elements have placeholder text contained that when a user starts typing, the placeholder text disappears.

\begin{lstlisting}[language=HTML]
<input type="text" name="field1" placeholder="Full Name" />
<input type="email" name="field2" placeholder="Email" />
<input type="url" name="field3" placeholder="Website" />
<input type="password" id="password_id" placeholder="Password">
<input type="number" name="field4" placeholder="Website" />
<label for="someFile">File input</label>
<input type="file" id="fileUpload">
<textarea placeholder=" Your Message" 
onkeyup="adjust_textarea(this)"></textarea>
<input class="button button-primary" value="Send Message" />
\end{lstlisting}

Other features such as radio and check boxes see figure Figure.~\ref{fig:radio} on  page~\pageref{fig:radio} have specific id's so that only one option can be checked at any given time. Below is the two types: 

\begin{lstlisting}[language=HTML]
<input type="radio" id="radio01" name="radio" />
<label for="radio01"><span></span>Radio Button 1</label>
<input type="radio" id="radio02" name="radio" />
<label for="radio02"><span></span>Radio Button 2</label>
<label><input type="checkbox"> Check this box</label>
<label><input type="checkbox"> Check this box</label>
<label><input type="checkbox"> Check this box</label>
\end{lstlisting}


\newpage
\subsection*{960 grid, 12 column system}
\addcontentsline{toc}{subsection}{960 grid, 12 column system}
Taylor'd UI is based on a 960 grid system. The grid system of 12 grids. 12 columns is used as each column can be evenly divided by 960. This allows the end user to build a seamless experience from desktops to mobile devices.

The columns are also used to layout the content of a web page. When using columns, it is important to remember that the columns in the section you are using them in, always match up to 12, see Figure.~\ref{fig:columnlayout} on  page~\pageref{fig:columnlayout} for an example. 

The class row-fluid is used to make the layout more fluid across devices. Media queries are also used to ensure that Taylor'd UI components collapse when they need to. 

\begin{lstlisting}[language=HTML]
<div class="row-fluid">
          <div class="col12 showcasing_grid">.col12</div>
          <div class="row-fluid">
            <div class="col6 showcasing_grid">.col6</div>
            <div class="col6 showcasing_grid">.col6</div>
          </div>
          <div class="row-fluid">
            <div class="col4 showcasing_grid">.col4</div>
            <div class="col4 showcasing_grid">.col4</div>
            <div class="col4 showcasing_grid">.col4</div>
          </div>
          <div class="row-fluid">
            <div class="col8 showcasing_grid">.col8</div>
            <div class="col4 showcasing_grid">.col4</div>
          </div>
          <div class="row-fluid">
            <div class="col3 showcasing_grid">.col3</div>
            <div class="col3 showcasing_grid">.col3</div>
            <div class="col3 showcasing_grid">.col3</div>
            <div class="col3 showcasing_grid">.col3</div>
          </div>
          <div class="row-fluid">
            <div class="col7 showcasing_grid">.col7</div>
            <div class="col5 showcasing_grid">.col5</div>
          </div>
          <div class="row-fluid">
            <div class="col1 showcasing_grid">.col1</div>
            <div class="col10 showcasing_grid">.col10</div>
            <div class="col1 showcasing_grid">.col1</div>
          </div>
        </div>
\end{lstlisting}

